\chapter{Conclusion}
\label{chap:conclusion}
This work aimed to isolate the sterilizing effects of radiation in APP treatment on C. sphaerospermum. To summarize, this chapter will provide an overview and an assessment of the results that were found. It will also discuss the limitations of the current work and suggest possible outlooks for future work.

The results of the experiments show that the subject of this study is a relevant topic when trying to understand the effects of APP sterilization further and the research question posed were answered. While not perfectly aligning the wavelengths, the two experiments leading up to the full treatment still provided a good basis to support the results of the final experiment. It was very clearly confirmed that radiation in the UV range around 250 nm is capable of sterilizing the spores of C. sphaerospermum and the control experiment concluded that its effects on the agar media are negligible at the doses used. It was found that an estimated dose of 0.35 mJ/cm² is sufficient to sterilize the spores of C. sphaerospermum in this setting. The plasma used in the experiment was found to emit UV radiation mainly in the range of 300 nm to 400 nm. The main species that was found in the OES measurements was Nitrogen and the radiation was primarily attributed to its 2PS. While not very conclusive, a Boltzmann plot was used and found the electron temperature to be around 6500 K or roughly 0.5 eV.

The use of the UV transparent quartz glass enabled a simple experiment, which was able to show that the radiation emitted by the APP alone is capable of sterilizing the spores and might play an important role. While the data is very limited and the number of results are not statistically conclusive, the concept has been proven and a deactivation of approximately 50 \% of spores was achieved after 30 minutes. The inclusion of a control experiment that estimates the hydroxyl radical concentration provided support for the assumption that the radiation is the main contributor to the sterilization in the experiment. Gaining a better understanding for the mechanisms, like radiation, that cause APP to be effective at sterilization is crucial for the development and improvement of APP as a sterilization method. It shows potential for treatment of surfaces without touching them directly, which could be a great advantage in.

To improve the results, the concentration of hydroxyl radicals should be lowered to a level that is comparable to that of the control group. This would allow for a more direct comparison of the two experiments and would help to quantify the role of the radiation in the sterilization process. For that a better seal of the quartz glass is needed to prevent the escape of hydroxyl radicals. Also, the possibility of hydroxyl radicals being formed behind the quartz glass by the UV radiation should be considered and investigated. For that a UV blocking layer could be applied to the quartz glass so that the only source of hydroxyl radicals is the APP itself. To further categorize the plasma additional experiments should be conducted and an absolute quantification of the radiation should be performed. As discussed in section \ref{sec:oes_temperature} Akatsuka and others \cite{oes_temperature} have presented a non-intrusive method that could be used to measure the electron temperature and density of the plasma accurately. Additionally, an experiment with a UV lamp capable of emitting the same wavelengths as the APP could be performed to establish a more direct comparison.

Because of the limited time that the author spent at KIT, it was not possible to conduct any more of the proposed experiments. However, the results of the current work seem promising and may lead to further research in this area in KIT's lab.