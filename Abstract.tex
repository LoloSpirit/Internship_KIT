\chapter*{Abstract}
Using atmospheric pressure plasma for surface sterilization has been shown to be a promising alternative to conventional chemical and thermal methods. While the focus of most studies has been on the inactivation of bacteria, this work explores the interaction of atmospheric pressure air plasma with a fungus, Cladosporium sphaerospermum, a mould known for to be resilient. The aim of the study is to investigate the role of radiation emitted by the plasma in the sterilization process and to get a better understanding of the mechanisms that lead to inactivation.

To achieve this, three approaches are taken: optical emission spectroscopy (OES) analysis of the plasma, ultraviolet (UV) exposure experiments using a UV lamp, and plasma treatment with its radiation isolated through a quartz glass barrier. The OES results identified the Second Positive System (2PS) of molecular nitrogen as the main source of UV emission. The wavelengths emitted are primarily between 300 nm and 400 nm. An approximate electron temperature of 6500 K was calculated using a Boltzmann plot. The exposure experiments showed that a UV dose of 0.35 mJ/cm² effectively inactivated C. sphaerospermum spores, while the plasma treatment with the quartz barrier achieved a 50\% inactivation rate after 30 minutes. This proves that the radiation emitted by the APP significantly contributes to the sterilization process. However, the effects of the hydroxyl radicals could not be fully isolated with the setup, the results suggest that the radiation plays a significant role in the inactivation of spores, but further research is needed to quantify its contribution accurately.