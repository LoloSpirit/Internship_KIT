\chapter{Introduction}
\label{chap:intro}
In various industries, including medical, agricultural fields and aerospace fields, the sterilization of surfaces or equipment is an ongoing challenge that still shows room for improvement. Many sterilization methods rely on the use of chemicals or high temperatures around 120 \textdegree C. This can be problematic as it can damage or fatigue materials and leave traces of toxic gases like formaldehyde \cite{app_study}. Using a plasma instead of chemicals or heat has been shown to be a promising alternative. Especially the use of atmospheric pressure plasmas (APP) is a great general-purpose sterilization method as the plasma can be generated without the need for a vacuum chamber. APPs are generally non-thermal, can be created around room temperature and don't need to be in direct contact with the materials they interact with. This allows for the treatment of sensitive materials and fine adjustment of the intensity by varying distance and time of exposure. 

In the past the treatment of bacteria with APPs has been the focus of many studies \cite{app_study,bacteria}, however another group of microorganisms, fungi, which include moulds, has not been studied as much. Moulds are known to be more resistant to sterilization than bacteria and are ubiquitous in virtually all environments. Many of the conventional methods fall short of effectively eradicating the spores of moulds especially in porous surfaces. This is a problem in the food industry, where moulds can spoil food and cause allergic reactions in humans but also in households \cite{mould, growth}.

It is known that the sterilization effect that APPs have on microorganisms is related primarily to reactive species that are formed in the plasma. Studies have also proposed that the radiation emitted by the plasma contributes to this, but it is not clear how much of the effect can be attributed to the radiation. The aim of this work is to investigate the effect of APPs on the mould Cladosporium sphaerospermum and to dissect the role of the radiation, specifically in the ultraviolet (UV) range, in the sterilization process. To research this the study tries to answer the following three questions:
\begin{itemize}
    \item Can UV radiation effectively inactivate spores of Cladosporium sphaerospermum?
    \item Does the APP used in this experiment emit UV at wavelengths known to be effective in spore inactivation?
    \item Does plasma-emitted radiation alone produce a sterilization effect on Cladosporium sphaerospermum spores, and how much of the overall effect can be attributed to this radiation?
\end{itemize}

To answer these questions multiple experiments are designed and conducted during a 4-week internship at the Kyoto Institute of Technology (KIT) in Japan. They isolate the plasma and the spores at first in two experiments that measure the inactivation of spores by UV radiation and a measurement of the optical spectrum of the plasma. Then the spores are treated with the APP and the effect of the radiation is isolated by using a transparent quartz glass barrier between the plasma and the spores. The results are then compared to the inactivation without the barrier to determine the effect of the radiation. Finally, they are analysed and discussed in the context of prior studies.

Because the mould needs multiple days of incubation time more literature research was done in parallel. The next chapter provides scientific context on the use of plasmas for sterilization, followed by some theory on the mould Cladosporium sphaerospermum and a direct connection to the author's university research on space applications.