\chapter{Theory and Methods}
\label{chap:theory}

In this chapter the theoretical background and the methods used in this work are presented. This includes the collection of research on the fungal species Cladosporium sphaerospermum, as well as the atmospheric pressure plasma and the different processes that lead to the sterilization of the mould as they interact. 

\section{Cladosporium sphaerospermum}
Cladosporium sphaerospermum (C. sphaerospermum) is a species of fungus. Fungi are a large group of eukaryotic organisms\footnote{Organisms whose cells have a membrane around their nucleus, which includes animals, plants and fungi.}  that include yeasts, moulds and common mushrooms. It is a black mould that is commonly found in indoor environments, especially in areas that are damp or have high humidity. It was first described by a German mycologist, Albert Julius Otto Penzig, in 1886 from decaying citrus plant material in Italy \cite{mould}. C. sphaerospermum primarily reproduces asexually through the production of conidia, which are non-motile spores\footnote{Spores not capable of movement.} formed at the tips of hyphae\footnote{Long, branching structures that grow from their tips. They give moulds their furry appearance.}. These spores are easily dispersed through the air, allowing the mould to rapidly spread into new environments. Moulds like C. sphaerospermum require humid conditions because moisture is essential for spore germination\footnote{The process by which a spore begins to grow new hyphae.}. In dry environments, spores usually remain dormant \cite{growth}.

\subsection{Growth and Morphology\footnote{The study of the structure of organisms.}}
C. sphaerospermum has a darkly-pigmented mycelium\footnote{A network of branching hyphae.} that can appear black or dark green. The colonies of the mould are typically flat and have more of a powdery appearance than other moulds. It is typical for fungi of the Cladosporium family to have branching, tree-like hyphae on whose ends conidia are formed in chains.

In addition to a humid environment optimal conditions for the growth of C. sphaerospermum include a temperature of 25  \textdegree C. It is however a  psychrophilic fungus\footnote{An organism that is able to grow in very low temperatures.} and can grow at temperatures as low as -5 \textdegree C \cite{mould}. It nourishes through saprotrophic nutrition, which is the process of using decaying or dead organic matter as a source of nutrients. This is why it is commonly found in decaying plant material, where it was also discovered. The conversion of starch, cellulose and other compounds such as carbon dioxide provides the energy needed for growth.

\subsection{Ecological Role and Habitat}
Like many fungi C. sphaerospermum plays an important role in the ecosystem as a decomposer. It breaks down dead organic matter, recycling nutrients back into the soil, where it thrives. This process is important as it fertilizes the soil and allows for the growth of new plants, enabling the life cycle. Because of its resilience to different conditions it is able to grow in many different environments which include anthropogenic places like inside of buildings. In humid areas and on porous surfaces, such as wood or concrete walls it can build mycelium and produce new spores. The easily dispersed spores 

While fungi, like C. sphaerospermum, are able to bind carbon dioxide from the atmosphere they are not a feasable option for carbon capture or conversion at larger scales.
\subsection{Effects on Human Health}
\subsection{Response to Radiation}

\section{Atmospheric Pressure Plasma}
\section{Optical Emission Spectroscopy}
\section{Dielectric Barrier Discharge}

\section{Sterilization Mechanism}
\subsection{Reactive Species}
\subsection{Radiation Effects}