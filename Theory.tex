\chapter{Theory and Methods}
\label{chap:theory}

In this chapter the theoretical background and the methods used in this work are presented. This includes the collection of research on the fungal species Cladosporium sphaerospermum, as well as the atmospheric pressure plasma and the different processes that lead to the sterilization of the mould as they interact. 

\section{Cladosporium sphaerospermum}
Cladosporium sphaerospermum (C. sphaerospermum) is a species of fungus. Fungi are a large group of eukaryotic organisms\footnote{Organisms whose cells have a membrane around their nucleus, which includes animals, plants and fungi.}  that include yeasts, moulds and common mushrooms. It is a black mould that is commonly found in indoor environments, especially in areas that are damp or have high humidity. It was first described by a German mycologist, Albert Julius Otto Penzig, in 1886 from decaying citrus plant material in Italy \cite{mould}. C. sphaerospermum primarily reproduces asexually through the production of conidia, which are non-motile spores\footnote{Spores not capable of movement.} formed at the tips of hyphae\footnote{Long, branching structures that grow from their tips. They give moulds their furry appearance.}. These spores are easily dispersed through the air, allowing the mould to rapidly spread into new environments. Moulds like C. sphaerospermum require humid conditions because moisture is essential for spore germination\footnote{The process by which a spore begins to grow new hyphae.}. In dry environments, spores usually remain dormant \cite{growth}.

\subsection{Growth and Morphology\footnote{The study of the structure of organisms.}}
C. sphaerospermum has a darkly-pigmented mycelium\footnote{A network of branching hyphae.} that can appear black or dark green. The colonies of the mould are typically flat and have more of a powdery appearance than other moulds. It is typical for fungi of the Cladosporium family to have branching, tree-like hyphae on whose ends conidia are formed in chains. The spores themselves are round to oval in shape and measure a few micrometers in length. They are very resilient and can stay alive even in conditions not favourable for growth. Due to their small size they are invisible to the naked eye.

In addition to a humid environment optimal conditions for the growth of C. sphaerospermum include a temperature of 25  \textdegree C. It is however a  psychrophilic fungus\footnote{An organism that is able to grow in very low temperatures.} and can grow at temperatures as low as -5 \textdegree C \cite{mould}. It nourishes through saprotrophic nutrition, which is the process of using decaying or dead organic matter as a source of nutrients. This is why it is commonly found in decaying plant material, where it was also discovered. The conversion of starch, cellulose and other compounds such as carbon dioxide provides the energy needed for growth.

\subsection{Ecological Role and Habitat}
Like many fungi C. sphaerospermum plays an important role in the ecosystem as a decomposer. It breaks down dead organic matter, recycling nutrients back into the soil, where it thrives. This process is important as it fertilizes the soil and allows for the growth of new plants, enabling the life cycle. Because of its resilience to different conditions it is able to grow in many different environments which include anthropogenic places like indoor environments. In humid areas and on porous surfaces, such as wood or concrete walls it can build mycelium and produce new spores. The easily dispersed spores reach virtually everywhere and are even found in orbiting spacecraft \cite{radiation}.

While fungi don't perform photosynthesis and therefore do not convert CO$_2$ into oxygen they still contribute to the carbon cycle. By digesting plant material and binding carbon dioxide in their biomass they play an important role in storing carbon and reducing the amount of carbon dioxide released into the atmosphere when plants decay. Because of there abundance they are able to store large amounts and help build stable compounds for long-term soil carbon storage.

\subsection{Effects on Human Health}
Because of its ubiquity C. sphaerospermum surrounds humans in their daily lives. The small size of the spores allows them to be inhaled deeply into the lungs where they can lead to allergenic reactions as well as pathogenic infections\footnote{A pathogen is any organism that can cause disease in a host.}, particularly when the immune system is weakened. Compared to other airborne moulds it is however not considered a serious health risk as it does not produce mycotoxins\footnote{Toxins produced by certain fungi that can cause severe disease in humans and animals.}. Fungi in general can cause respiratory problems, especially when inhaling them in large quantities over a long time. They are however also essential in medical science as they are used to produce antibiotics, such as penicillin. While working with them in the lab clean benches can be used to provide sufficient ventilation but they generally don't pose any danger to a healthy person.

\subsection{Response to Radiation}
UV radiation is commonly used to sterilize surfaces and is very effective in killing most microorganisms like bacteria and fungi. C. sphaerospermum however shows a higher resistance to different types of radiation including UV and even high energy ionizing radiation.  
It is able to not only survive doses of ionizing radiation that would kill most other organisms but also to thrive in these conditions. This was discovered in 1990s as the fungus is able to grow in the high radiation environment of the destroyed Chernobyl nuclear power plant \cite{radiation}, even on the highly radioactive reactor walls. The mould does not only survive in these conditions but was proposed to be radiotrophic - a process where fungi potentially use ionizing radiation as a source of energy analogous to photosynthesis - growing faster.

Its tolerance to radiation both ionizing and non-ionizing is attributed to the pigment molecule Melanin. Pigment molecules such as Melanin absorb visible light and other radiation which can protect cells, can be used to absorb energy and also gives the organism its colour. It gives the mould its dark colour and protects its from UV radiation. Melanin is also able to absorb ionizing radiation and potentially convert it into chemical energy usable to the fungus.

To further research this discovery the mould was sent to the International Space Station in 2018 where it was exposed to cosmic radiation \cite{iss}. The study found that the mould was able to survive the radiation and even thrive in the microgravity environment while absorbing some of the energy. The results of the study suggest that C. sphaerospermum could potentially serve as a highly sought-after solution for creating a self-replicating biological radiation shield or be otherwise useful in space exploration.

\section{Atmospheric Pressure Plasma}
\section{Optical Emission Spectroscopy}
\section{Dielectric Barrier Discharge}

\section{Sterilization Mechanism}
\subsection{Reactive Species}
\subsection{Radiation Effects}